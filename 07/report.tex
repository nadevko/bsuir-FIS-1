\documentclass{bsuir}
\usepackage{makecell}
\usepackage{multirow}
\usepackage{amsmath}
\usepackage{breqn}

\departmentlong{инженерной психологии и эргономики}
\worktitle{\textbf{ОТЧЁТ}\\к практическому занятию на тему\\\textquote{Оценка стойкости парольной защиты данных}}
\titleleft{
    Проверил:\\
    Фомин Д.А.\\
    ~
}
\titleright{
    Выполнили:\\
    Дриц М.Ф.\\
    Бородин А.Н.\\
    гр. 310901
}
\titlepageyear{2025}

\renewcommand*{\thesection}{\arabic{section}}

\begin{document}

\maketitle
\mainmatter

\textbf{Цель работы:} оценить стойкость различных типов паролей и определить минимальные параметры, обеспечивающие защиту информации в течение заданного времени.

\textbf{Исходные данные:}

\begin{itemize}
	\item вариант: 7;
	\item скорость перебора: $V = 100$ паролей/сек;
	\item время атаки: $T = 3$ месяца = 90 дней;
	\item допустимое количество перебираемых паролей: $P = 10^{-2}$.
\end{itemize}

\[
	\begin{aligned}
		V   & = 100~\text{пар/сек} = 6000~\text{пар/мин} = 360000~\text{пар/час} = 8640000~\text{пар/день} \\
		S^* & = \frac{V \cdot T}{P} = \frac{8640000 \cdot 90}{10^{-2}} = 77760000000 = 7{,}776 \cdot 10^7
	\end{aligned}
\]

\section{Оценка минимальной длины пароля}

Для алфавита $A = 26$ (малые латинские буквы):

\[
	\begin{aligned}
		S              & = A^L \geq 7{,}776 \cdot 10^7                                                     \\
		\log_{10}(A^L) & \geq \log_{10}(7{,}776 \cdot 10^7) \Rightarrow L \cdot \log_{10}(26) \geq 10{,}89 \\
		L              & \geq \frac{10{,}89}{\log_{10}(26)} \approx \frac{10{,}89}{1{,}415} \approx 7{,}7
	\end{aligned}
\]

\textbf{Вывод:} минимальная длина пароля — 8 символов при использовании латинского алфавита.

\section{Время подбора паролей}

\begin{itemize}
	\item Малые латинские буквы (8 символов): \textbf{52 сек}
	\item Одинаковые латинские буквы (8 символов): \textbf{2 сек}
	\item Малые русские буквы (8 символов): \textbf{352 сек}
	\item Одинаковые русские буквы (8 символов): \textbf{11 сек}
\end{itemize}

\textbf{Вывод:} пароли, содержащие повторяющиеся символы, подбираются значительно быстрее. Использование различных символов из большого алфавита значительно повышает стойкость.

\section{Минимальная длина пароля для стойкости не менее одного года}

\begin{itemize}
	\item Русский алфавит (33 символа): \textbf{10 символов}
	\item Латинский алфавит (26 символов): \textbf{12 символов}
\end{itemize}

\section{Примеры стойких паролей}

\begin{itemize}
	\item Латинский алфавит: \texttt{P@bg95b!N} (9 символов)
	\item Русский алфавит: \texttt{П.@к0ль!} (8 символов)
\end{itemize}

\textbf{Вывод:} комбинирование символов разного типа (буквы, цифры, спецсимволы) позволяет создавать короткие, но стойкие пароли.

\end{document}
