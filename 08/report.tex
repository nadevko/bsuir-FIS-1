\documentclass{bsuir}
\usepackage{makecell}
\usepackage{multirow}

\departmentlong{инженерной психологии и эргономики}
\worktitle{\textbf{ОТЧЁТ}\\к практическому занятию на тему\\\textquote{Оценка рисков информационной безопасности}}
\titleleft{
    Проверил:\\
    Фомин Д.А.\\
    ~\\
    ~
}
\titleright{
    Выполнили:\\
    Бородин А.Н.\\
    Дриц М.Ф.\\
    гр. 310901
}
\titlepageyear{2025}

\usepackage{pgfplots}
\usepackage{amsmath}
\usepackage{breqn}

\newlength{\tablewidth}
\setlength{\tablewidth}{\textwidth - \parindent}

\renewcommand*{\thesection}{\arabic{section}}

\begin{document}

\maketitle
\mainmatter

\textbf{Цель работы:} изучить методику оценки рисков информационной безопасности и получить
практические навыки по её применению.

\textbf{Вариант:} архитектура информационной системы, изображённая на рисунке \ref{img:variant.png}.

\makeimage[Архитектура информационной системы]{variant.png}

\section{Цена ущерба при реализации угрозы 1}

Процент повреждений активов в результате реализации угрозы 1 (6 раз за год, вероятность $p_1 = 0{,}6$):

\begin{itemize}
	\item актив 1 (7 тыс. руб.): 100\%;
	\item актив 2 (5 тыс. руб.): 100\%;
	\item актив 3 (32 тыс. руб.): 100\%;
	\item актив 4 (1000 тыс. руб.): 30\%;
	\item актив 5 (50000 тыс. руб.): 0\% (ущербом можно пренебречь);
	\item стоимость простоя: 21 тыс. руб. за год.
\end{itemize}

Ущерб за один инцидент:
$$U_{1~\text{акт}} = 7 + 5 + 32 + 1000 \times 0{,}3 = 44 + 300 = 344\text{ тыс.~руб}.$$

Суммарный ущерб:
$$U_1 = 6 \times 344 + 21 = 2064 + 21 = 2085\text{ тыс.~руб}.$$

\section{Цена ущерба при реализации угрозы 2}

Величины ущерба от реализации угрозы 2 (1 раз за год, $p_2 = 0{,}4$):

\begin{itemize}
	\item Финансовые потери от разглашения информации: 56 тыс. руб.;
	\item Ущерб репутации: 88 тыс. руб.
\end{itemize}

Суммарный ущерб:
$$U_2 = 56 + 88 = 144\text{ тыс.~руб}.$$

\section{Общий риск реализации угроз}

Для угрозы~1:
$$R_1 = p_1 \times U_1 = 0{,}6 \times 2085 = 1251\text{ тыс.~руб}.$$

Для угрозы~2:
$$R_2 = p_2 \times U_2 = 0{,}4 \times 144 = 57{,}6\text{ тыс.~руб}.$$

Общий риск:
$$R_{\text{общ}} = R_1 + R_2 = 1251 + 57{,}6 = 1308{,}6\text{ тыс.~руб}.$$

\section{Оптимальное распределение бюджета}

\begin{itemize}
	\item Годовой бюджет на меры безопасности: 80 тыс. руб.;
	\item Стоимость антивируса: 90 тыс. руб.;
	\item Стоимость системы парольной защиты: 20 тыс. руб.;
	\item Общая стоимость мер безопасности: $90 + 20 = 110$ тыс. руб.;
	\item Дефицит бюджета: $110 - 80 = 30$ тыс. руб.
\end{itemize}

Пусть $x$ "--- процент сокращения бюджета на антивирус, а $y$ "--- процент сокращения бюджета на систему парольной
защиты.

Дополнительный остаточный риск и экономия от сокращения бюджета:
$$R_A(x) = 584.4 \times \frac{x}{100}, E_A = 90 \times \frac{x}{100} = 0.9x$$
$$R_P(y) = 141.6 \times \frac{y}{100}, E_P = 20 \times \frac{y}{100} = 0.2y$$

Суммарная экономия должна быть равна дефициту бюджета:
$$0.9x + 0.2y = 30$$

\section{Решение задачи оптимизации}

Целевая функция (минимизация общего дополнительного остаточного риска):
$$Z = R_A(x) + R_P(y) = 5.844x + 1.416y$$

Ограничения на переменные: $0 \le x \le 100, 0 \le y \le 100$

Выразим $y$ через $x$ из бюджетного ограничения:
$$0.2y = 30 - 0.9x$$
$$y = \frac{30 - 0.9x}{0.2} = 150 - 4.5x$$

Подставим это выражение для $y$ в целевую функцию $Z$:
$$Z(x) = 5.844x + 1.416(150 - 4.5x)$$
$$Z(x) = 5.844x + 212.4 - 6.372x$$
$$Z(x) = 212.4 - 0.528x$$

Для минимизации $Z(x)$ необходимо максимизировать $x$ (так как коэффициент перед $x$ отрицательный). Рассмотрим
ограничения на $x$:

\begin{itemize}
	\item $x \le 100$ (максимальное сокращение)
	\item $y \ge 0 \Rightarrow 150 - 4.5x \ge 0 \Rightarrow 150 \ge 4.5x \Rightarrow x \le \frac{150}{4.5} \approx 33.33$
\end{itemize}

Следовательно, мы должны сократить бюджет на антивирус на $33.3\%$ и не сокращать бюджет на парольную систему:
$$x = 33.33\%, y = 150 - 4.5 \times 33.33 = 150 - 149.985 \approx 0\%$$

Дополнительный риск при оптимальном распределении (антивирус):
$$R_A = 584.4 \times \frac{33.3}{100} = 584.4 \times 0.333 \approx 194.65 \text{ тыс. руб.}$$

Дополнительный риск при оптимальном распределении (парольная система):
$$R_P = 0 \text{ тыс. руб.}$$

Итоговый остаточный риск:
$$R_{\text{ост}} = R_A + R_P = 194.65 + 0 = 194.65 \text{ тыс. руб.}$$

Эффективность мер:
$$E = \frac{R_{\text{общ}} - R_{\text{ост}}}{R_{\text{общ}}} \times 100\% = \frac{1308.6 - 194.8}{1308.6} \times 100\% \approx 85.1\%$$

\section{Критичность и уровень угроз}

Потенциальный ущерб от реализации уязвимостей:
\begin{itemize}
	\item $ER_{1,1} = 100\%$ (угроза 1, уязвимость 1)
	\item $ER_{1,2} = 20\%$ (угроза 1, уязвимость 2)
	\item $ER_{2,1} = 40\%$ (угроза 2, уязвимость 1)
	\item $ER_{2,2} = 30\%$ (угроза 2, уязвимость 2)
\end{itemize}

Уровень угрозы по уязвимости ($Th_{i,j} = ER_{i,j} \times P(V)$):
\begin{itemize}
	\item $Th_{1,1} = 100\% \times 0.5 = 50\%$
	\item $Th_{1,2} = 20\% \times 0.5 = 10\%$
	\item $Th_{2,1} = 40\% \times 0.5 = 20\%$
	\item $Th_{2,2} = 30\% \times 0.5 = 15\%$
\end{itemize}

Суммарный уровень угроз:
$$CTh_i = 1 - \prod_{j=1}^{n} (1 - Th_{i,j})$$

Для угрозы 1:
$$CTh_1 = 1 - (1 - 0.50) \times (1 - 0.10) = 1 - 0.45 = 0.55 = 55\%$$

Для угрозы 2:
$$CTh_2 = 1 - (1 - 0.20) \times (1 - 0.15) = 1 - 0.68 = 0.32 = 32\%$$

\end{document}