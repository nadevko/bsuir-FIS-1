\documentclass{bsuir}
\usepackage{makecell}
\usepackage{multirow}

\departmentlong{инженерной психологии и эргономики}
\worktitle{\textbf{ОТЧЁТ}\\к практическому занятию на тему\\\textquote{Оценка рисков информационной безопасности}}
\titleleft{
    Проверил:\\
    Фомин Д.А.\\
    ~\\
    ~
}
\titleright{
    Выполнили:\\
    Бородин А.Н.\\
    Дриц М.Ф.\\
    гр. 310901
}
\titlepageyear{2025}

\usepackage{pgfplots}
\usepackage{amsmath}
\usepackage{breqn}

\newlength{\tablewidth}
\setlength{\tablewidth}{\textwidth - \parindent}

\renewcommand*{\thesection}{\arabic{section}}

\begin{document}

\maketitle
\mainmatter

\textbf{Цель работы:} изучить методику оценки рисков информационной безопасности и получить
практические навыки по ее применению.

\textbf{Вариант:} архитектура информационной системы.

\section{Найти цену ущерба при реализации угрозы 1}

Рассчитываем цену ущерба по:

\begin{itemize}
    \item актив 1: 7 тыс. руб. (поврежден на 100\%);
    \item актив 2: 5 тыс. руб. (поврежден на 100\%);
    \item актив 3: 32 тыс. руб. (поврежден на 100\%);
    \item актив 4: 1000 тыс. руб. (поврежден на 30\%);
    \item ущерб от нарушения деятельности: 21 тыс. руб.
\end{itemize}

\[
(7+5+32) \times 1 + 1000 \times 0.3 + 21 = 44 + 300 + 21 = 365 \text{ тыс. руб.}
\]

\section{Найти цену ущерба при реализации угрозы 2}

Рассчитываем цену ущерба по:

\begin{itemize}
    \item Финансовые потери от передачи информации третьим лицам: 56 тыс. руб.
    \item Ущерб репутации: 88 тыс. руб.
\end{itemize}

\[
56 + 88 = 144 \text{ тыс. руб.}
\]

\section{Найти общий риск при реализации угрозы 1 и угрозы 2}

\begin{align*}
\text{РИСК 1} = 0.6 \times 365 = 219 \text{ тыс. руб.} \\
\text{РИСК 2}= 0.4 \times 144 = 57.6 \text{ тыс. руб.}\\
219 + 57.6 = 276.6 \text{ тыс. руб.}
\end{align*}

\section{Оптимальное распределение средств}

Выделенный бюджет: 80 тыс. руб.

\begin{itemize}
    \item Пусть $x\%$ -- сокращение бюджета антивируса и $y\%$ -- сокращение бюджета парольной системы.
    \item Стоимость антивируса: 90 тыс. руб.
    \item Стоимость парольной системы: 20 тыс. руб.
    \item Формулы остаточного риска:
    \begin{align*}
    R(A) &= \frac{584.4}{100} \times x \\
    R(P) &= \frac{141.6}{100} \times y
    \end{align*}
\end{itemize}

Решая уравнения, получаем оптимальные $x$ и $y$ для минимального остаточного риска:

\begin{align*}
x &= 11.1\% \\
y &= 0\% \text{ (парольная защита остается полной)}
\end{align*}

Подставляем в формулы остаточного риска:

\begin{align*}
R(A) = \frac{584.4}{100} \times 11.1 = 64.92 \text{ тыс. руб.} \\
R(P) = \frac{141.6}{100} \times 0 = 0 \text{ тыс. руб.}\\
276.6 - 64.92 - 0 = 211.68 \text{ тыс. руб.}
\end{align*}

\section{Эффективность мер}

\begin{align*}
E = \frac{276.6 - \text{Остаточный Риск}}{276.6} \times 100\%
= \frac{276.6 - 211.68}{276.6} \times 100\% = 23.48\%
\end{align*}

\section{Определение критичности и уровня угроз}

\begin{align*}
Th_1 &= \frac{100 \times 60}{100} = 60 \\
Th_2 &= \frac{100 \times 40}{100} = 40 \\
CTh &= 1 - (1 - \frac{Th_1}{100}) \times (1 - \frac{Th_2}{100}) \\
CTh &= 1 - (1 - \frac{60}{100}) \times (1 - \frac{40}{100})
= 1 - (0.4 \times 0.6)
= 76\%
\end{align*}

\end{document}
