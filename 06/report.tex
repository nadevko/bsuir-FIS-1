\documentclass{bsuir}
\usepackage{makecell}
\usepackage{multirow}
\usepackage{amsmath}
\usepackage{breqn}

\departmentlong{инженерной психологии и эргономики}
\worktitle{\textbf{ОТЧЁТ}\\к практическому занятию на тему\\\textquote{Определение признаков фишинга по содержанию
\\сообщений электронной почты}}
\titleleft{
    Проверил:\\
    Фомин Д.А.\\
    ~
}
\titleright{
    Выполнили:\\
    Дриц М.Ф.\\
    Бородин А.Н.\\
    гр. 310901
}
\titlepageyear{2025}

\renewcommand*{\thesection}{\arabic{section}}

\begin{document}

\maketitle
\mainmatter

\textbf{Цель работы:} изучить признаки фишинга содержащихся в сообщениях электронной почты и
получить практические навыки их обнаружения в таких сообщениях.

\textbf{Исходные данные:} изображены на рисунке \ref{img:variant.png}.

\makeimage[Изучаемое письмо]{variant.png}[width=.9\textwidth]

\section{Анализ фишингового сообщения}

Заметны следующие признаки фишинга:

\begin{enumerate}
	\item \textbf{Подозрительный адрес отправителя:} письмо пришло с \url{smipfox-x6e8g@bes-marine.com} — домен, не связанный с известными почтовыми службами или официальными организациями.

	\item \textbf{Время отправления:} 24 мая 2021 года, 06:11 — нерабочее время, что является типичным для массовых фишинговых рассылок.

	\item \textbf{Отсутствие персонализации:} в письме нет обращения по имени или других персонализированных элементов.

	\item \textbf{Лингвистические ошибки:} наличие некорректных слов (например, «unamcogured») указывает на непрофессиональный характер сообщения.

	\item \textbf{Эмоциональное давление:} создание чувства срочности и угроза блокировки аккаунта — типичная тактика манипуляции.

	\item \textbf{Подозрительная ссылка:} кнопка «Verify Now» ведет на внешний ресурс, вероятно, созданный для кражи учетных данных.
\end{enumerate}

\textbf{Вывод:} данное письмо является классическим примером фишинговой атаки, направленной на получение доступа к личным данным пользователя через манипуляцию и создание ложного чувства срочности. Рекомендуется удалить такое письмо и никогда не переходить по содержащимся в нем ссылкам.

\section{Создание фишингового сообщения с минимальными признаками фишинга}

Для создания фишингового сообщения с минимальными признаками использованы следующие приемы:
персонализированное обращение по имени получателя, что повышает доверие; отправка с домена,
визуально очень похожего на официальный корпоративный (steampoweved.com вместо steampowered.com);
профессиональное оформление текста в стиле стандартного уведомления о безопасности без излишней
срочности; и размещение ссылки, которая внешне напоминает официальный сайт компании, но ведет на
фишинговую копию. Такое сообщение значительно сложнее идентифицировать как мошенническое из-за его
профессионального вида и отсутствия явных признаков фишинга.

\begin{quote}
Уважаемый SuperDarkStalker2007Ultra,

Адрес электронной почты, связанный с вашим аккаунтом Steam, был успешно изменён.

Мы отправляем это уведомление для обеспечения безопасности вашего аккаунта. Если вы подтвердили это изменение, никаких дополнительных действий предпринимать не нужно.

Если вы не авторизовали это изменение, выполненное с компьютера, расположенного по IP"=адресу 243.232.222.111 (Республика Корея), пожалуйста, немедленно измените ваш пароль от Steam, а также рекомендуется изменить пароль от вашей электронной почты для дополнительной безопасности.

Если вы не можете получить доступ к своему аккаунту, вы можете использовать эту одноразовую \url{ссылку} для восстановления или блокировки аккаунта.

С уважением,
Команда Steam
\end{quote}

\end{document}
