\documentclass{bsuir}
\usepackage{makecell}
\usepackage{multirow}

\departmentlong{инженерной психологии и эргономики}
\worktitle{\textbf{ОТЧЁТ}\\к практическому занятию на тему\\\textquote{Анализ демаскирующих признаков объектов}}
\titleleft{
    Проверил:\\
    Фомин Д.А.\\
    ~
}
\titleright{
    Выполнили:\\
    Бородин А.Н.\\
    гр. 310901
}
\titlepageyear{2025}

\usepackage{pgfplots}
\usepackage{amsmath}
\usepackage{breqn}

\newlength{\tablewidth}
\setlength{\tablewidth}{\textwidth - \parindent}

\renewcommand*{\thesection}{\arabic{section}}

\begin{document}

\maketitle
\mainmatter

\textbf{Цель работы:} идентифицировать объект по его демаскирующим признакам.

\textbf{Вариант:} изображён на рисунке \ref{img:variant.png}.

\makeimage[Идентифицируемый объект]{variant.png}[width=.4\textwidth]

\section{Совокупность демаскирующих признаков}

\textbf{Опознавательные признаки:}

\begin{itemize}
	\item прямоугольная форма (видовой / прямой / постоянный);
	\item твёрдая обложка с отчетливо видимым корешком (видовой / прямой / постоянный);
	\item наличие блока страниц, характерного для печатного издания (видовой / прямой / постоянный);
	\item название \textquote{Незнайка на Луне} (видовой / именной / постоянный);
	\item имя автора \textquote{Н. Носов} (видовой / именной / постоянный);
	\item логотип или наименование издательства \textquote{Детская литература} (видовой / именной / постоянный);
	\item иллюстрация на обложке, изображающая луну и персонажа (Незнайку) (видовой / прямой / постоянный);
	\item тёмно"=синяя и жёлтая цветовая окраска (видовой / прямой / постоянный);
	\item шрифтовое оформление заголовка и текстовых элементов (видовой / прямой / постоянный);
	\item матовая текстура поверхности обложки (вещественный / прямой / постоянный).
\end{itemize}

\textbf{Признаки деятельности:}

\begin{itemize}
	\item возможность открытия и закрытия книги (видовой / прямой / периодический);
	\item необходимость перелистывания страниц при чтении (видовой / прямой / периодический);
	\item шелест страниц, сопровождающий перелистывание (сигнальный / косвенный / эпизодический);
	\item износ обложки и переплёта при длительном использовании (видовой / косвенный / постоянный);
	\item появление складок и заломов на страницах в процессе чтения (видовой / косвенный / эпизодический).
\end{itemize}

\section{Процедура идентификации}

\textbf{Определение класса объекта:}
По опознавательным признакам (прямоугольная форма, наличие блока страниц,
твёрдая обложка, стандартные размеры) делается вывод, что объект относится к
печатным изданиям, а именно к книгам.

\textbf{Уточнение принадлежности объекта:}
Наличие названия \textquote{Незнайка на Луне} на обложке, указание автора
\textquote{Н. Носов} и логотип издательства \textquote{Детская литература}
позволяют классифицировать книгу как детскую художественную литературу.
Иллюстрации, характерная цветовая гамма и композиционное оформление
дополнительно подтверждают жанровую принадлежность и целевую аудиторию издания.

\textbf{Идентификация объекта:}
Именные признаки (название книги, имя автора, наименование издательства)
однозначно выделяют данную книгу среди иных печатных изданий.

Таким образом, совокупность демаскирующих признаков позволяет однозначно
идентифицировать объект как книгу \textquote{Незнайка на Луне} Н. Носова.

\end{document}
