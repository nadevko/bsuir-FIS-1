\documentclass{bsuir}
\usepackage{makecell}
\usepackage{multirow}

\departmentlong{инженерной психологии и эргономики}
\worktitle{\textbf{ОТЧЁТ}\\к практическому занятию на тему\\\textquote{Оценка
достоверности информации}}
\titleleft{
    Проверил:\\
    Фомин Д.А.\\
    ~
}
\titleright{
    Выполнил:\\
    Бородин А.Н.\\
    гр. 310901
}
\titlepageyear{2025}

\usepackage{pgfplots}
\usepackage{amsmath}
\usepackage{breqn}

\newlength{\tablewidth}
\setlength{\tablewidth}{\textwidth - \parindent}

\renewcommand*{\thesubsection}{\arabic{subsection}}

\begin{document}

\maketitle
\mainmatter

\textbf{Цель работы:} проверить достоверность статьи \textquote{Microsoft
подтвердила, что отключит Skype 5 мая 2025 года с переводом пользователей в
Teams Free} (https://habr.com/ru/news/886818/).

\subsection{Автор}
Автор статьи "--- пользователь хабра Денис @denis-19, являющийся журналистом из
состава редакции портала с 2018 года и зарегистрированный с 2014 года, таким
образом, его статьи "--- одни из наиболее авторитетных на сайте и он входит в
число к ответственным за достоверность информации на хабре. Как автор, он
заслуживает доверие равное самой площадке. С ним возможно связаться через хабр
зарегистрированным пользователям, коим может стать любой желающий.

Он специализируется на новостях про IT"=компании, к которым относится статья, и
регулярно публикует новости о жизненном цикле ПО: новые релизы и закрытия
программ. Помимо новостей про компании, весомую долю в опубликованных им
новостях составляют тематики: \textquote{Научно-популярное},
\textquote{Информационная безопасность}, \textquote{Законодательство в IT},
\textquote{Финансы в IT}, \textquote{Социальные сети и сообщества},
\textquote{Космонавтика}, \textquote{Облачные сервисы}, \textquote{Будущее
здесь}, \textquote{IT"=инфраструктура}. Всего им опубликовано более 10 тысяч
новостей и сотни статей. Последние "--- преимущественно научно"=космической
тематики.

В статье он ссылается на авторитетное американское издание TechCrunch и
оффициальный блог, ютуб"=канал и X"=блог компании Microsoft, которая закрывает
Skype. Помимо этого упомяты статьи на хабре о всех предыдущих значимых событиях
в истории сервиса, его сайт и новость с ресурса XDA Developers, об обнаружение
возможного закрытия сервиса до каких либо заявлений компании об этом. Ссылки на
оффициальное сообщение, первоисточника, уже достаточно, чтобы гарантировать
достоверность.

\subsection{Анализ источника}
Хабр "--- независимое СМИ, коллективный блог с элементами новостного ресурса,
входящий в тройку лидеров русскоязычных IT"=изданий. Начал работать 26 мая 2006
года. Это старое (по меркам русскоязычной IT"=отрасли) и очень популярное
издание.

\subsection{Факты и оценки}
Факты:

\begin{itemize}
    \item
    Microsoft подтвердила, что отключит мессенджер Skype 5 мая 2025;
    \item
    Пользователи будут переведены на платформу Teams Free;
    \item
    В конце февраля 2025 года разработчики ресурса XDA Developers обнаружили в
    новой предварительной версии Skype информацию об этом;
    \item
    Microsoft прекратила поддержку классической версии Skype в ноябре 2018 года;
    \item
    В декабре 2024 года Microsoft прекратила обслуживание сервиса Skype Number;
    \item
    Летом 2024 года Microsoft возродила Skype;
    \item
    Skype впервые был запущен в 2003 году;
    \item
    Microsoft приобрела Skype в 2011 году за \$8,5 млрд.
\end{itemize}

Оценки:

\begin{itemize}
    \item
    Компания обновляла и переделывала Skype несколько раз, чтобы позиционировать
    его как жизнеспособного конкурента iMessage;
    \item
    В последние несколько лет Skype игнорировался пользователями;
    \item
    Microsoft не смогла позиционировать Skype как жизнеспособный вариант во
    время пандемии;
    \item
    Сервис в значительной степени отошёл на второй план в глазах потребителей;
    \item
    Большинство пользователей Skype перешли на более современные платформы,
    включая Telegram, Zoom, WhatsApp, iMessage и Discord.
\end{itemize}

\subsection{Полнота информации}
Информация полна и многократно доказана.

\end{document}
